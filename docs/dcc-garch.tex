% Этот шаблон документа разработан в 2014 году
% Данилом Фёдоровых (danil@fedorovykh.ru) 
% для использования в курсе 
% <<Документы и презентации в \LaTeX>>, записанном НИУ ВШЭ
% для Coursera.org: http://coursera.org/course/latex .
% Вы можете изменять, использовать, распространять
% этот документ любым способом по своему усмотрению. 
% В качестве благодарности автору вы можете сохранить 
% в начале документа данный текст или просто ссылку на
% http://coursera.org/course/latex
% Исходная версия Шаблона --- 
% https://www.writelatex.com/coursera/latex/1.1


\documentclass[a4paper,11pt]{article}

\usepackage{cmap}					% поиск в PDF
\usepackage[T2A]{fontenc}			% кодировка
\usepackage[utf8]{inputenc}			% кодировка исходного текста
\usepackage[english,russian]{babel}	% локализация и переносы
\usepackage{amsmath}	            % локализация и переносы
\setlength{\parindent}{1em}         % отступ в начале каждого параграфа
\usepackage{courier}                % шрифт для кода 
\usepackage{mathptmx}               % шрифт Times New Roman 
\usepackage{setspace}
%\полуторный интервал
\onehalfspacing
\usepackage{relsize}                % большие математические операторы

\begin{document} % Конец преамбулы, начало текста.

\section{Многомерные модели. Формулировка DCC-GARCH модели}%\label{razdel}

% R. F. Engle, Dynamic conditional correlation - a simple class of multivariate  garch models. UCSD, May 2001.

Существует несколько наиболее распространённых спецификаций моделей многомерной волатильности, особенности которых достаточно хорошо изучены в научной литературе: VECH, BEKK, DCC-GARCH. С учётом ограниченного размера набора данных, рассматриваемого в данной работе (810 торговых дней, когда известны котировки для всех четырёх акций), и необходимости проведения бэктеста для значений Value at Risk, рассчитанных по многомерной модели, нами было принято решение выбрать спецификацию для многомерной модели, которая бы потребовала оценки минимального числа параметров. Для четырёх активов без учёта модели среднего:
\begin{itemize}
    \item VECH требует оценить 210 параметров.
    \item BEKK требует оценить 42 параметра.
    \item DCC-GARCH необходимо оценить 15 параметров.
\end{itemize}

Исходя из сравнения, нами была выбрана DCC-GARCH модель. DCC-GARCH (Dinamic Conditional Correlation - GARCH) была предложена (Engle, 2001). В общем виде она может быть описана следующим образом:
\begin{equation}
\begin{aligned}
               r_t = \mu_t + \varepsilon_t \\
               \varepsilon_t = H^{1/2}_t z_t
\end{aligned}
\end{equation}

Где:
\begin{itemize}
    \item $r_t$ : $n \times 1$ вектор логарифмических доходнотсей $n$ активов в момент времени $t$.
    \item $\varepsilon_t$: $n \times 1$ вектор стандартизованных доходностей $n$ активов в момент времени $t$, т. е.
          \begin{enumerate}
                \item $E[\varepsilon_t]=0$
                \item $Var[\varepsilon_t] = Ht$
          \end{enumerate}
    \item $\mu_t$: $n \times 1$ вектор математических ожиданий условных доходностей $r_t$ (порождается моделью среднего для доходности).
    \item $H_t$: $n \times n$ матрица условной дисперсии в момент $t$.
    \item $H^{1/2}_t$ : Любая матрица размерности $n \times n$ в момент времени $t$, такая что $H_t$ - матрица условной дисперсии $\varepsilon_t$. $H^{1/2}_t$ может быть получена путём разложения Холецкого $H_t$.
    \item $z_t$: $n \times 1$ вектор независимых и одинаково распределённых случайных величин, таких что $E[z_t]=0$ и $E[z^{T}_t z_t] = I$.
\end{itemize}

Для модели DCC-GARCH изучены и реализованы следующие распределения случайной ошибки $\varepsilon_t$: 

\begin{enumerate}
    \item $\varepsilon_t \sim \mathrm{Normal}$.
    \item $\varepsilon_t \sim \mathrm{Student}$.
    \item $\varepsilon_t \sim \mathrm{Laplace}$.
\end{enumerate}

Модель среднего $\mu$ может быть выбрана ислледователем в зависимости от предполагаемых особенностей временного ряда; так, в статистических пакетах языка \texttt{R} реализована возможность оценки $AR, MA, ARIMA, ARFIMA$, а также константной модели среднего для DCC-GARCH.

Ключевой элемент DCC-GARCH - спецификация модели для $H_t$. Ковариационная матрица раскладывается на произведение матриц стандартных отклонений и корреляционной матрицы:
$$
H_t = D_t R_t D_t
$$
Ковариационная матрица $D_t$ диагональна, $d_{ii}$ представляют собой оценки стандартного отклонения, полученные с помощью одномерных GARCH(p, q) моделей для каждого актива в отдельности:
$$
\begin{gathered}

D_t = \begin{pmatrix} 
    \sqrt{h_{1t}} & 0 & \dots \\
    \vdots & \ddots & \\
    0 &  & \sqrt{h_{nt}} 
    \end{pmatrix}       \\[2ex]
%\setlength{\jot}{3}
h^{2}_{ii}= \omega + 
            \mathlarger{\mathlarger{\sum}}_{q=1}^{Q_i} \alpha_{qi} \varepsilon^{2}_{t-j} + 
            \mathlarger{\mathlarger{\sum}}_{p=1}^{P_i} \beta_{pi} h^{2}_{t-i}
  
\end{gathered}
$$

Корреляционная матрица $R$ оценивается для стандартизированных случайных шоков $\xi_t = D^{-1}_t \varepsilon_t \sim \mathrm{N}(0, R_t)$.Она положительно определена, $\rho_{ii}=1, i=1, ..., n$, и описывается уравнениями:
$$
\begin{gathered}
R_t = Q^{*-1}_t Q_t Q^{*-1}_t\\
Q_t = (1-a-b) \overline{Q} + a \xi_{t-1} \xi^{T}_{t-1} + b Q_{t-1} \\
\overline{Q} = \frac{1}{T} \mathlarger{\mathlarger{\sum}}_{t=1}^{T} \xi_t \xi^{T}_t
\end{gathered}
$$

Где $a$, $b$ - скаляры, матрица $\overline{Q}$, как следует из уравнений выше - просто оценка средней корреляции, $Q^{*-1}_t$ - матрица нормировки, которая приводит итоговые значения на диагонали оценённой корреляционной матрицы к единице. 



\subsection{Подраздел}\label{erty}

В разделе \ref{razdel} начинается документ.

Это подраздел \ref{erty}.


\end{document} % Конец текста.

